% Template for a Computer Science Tripos Part II project dissertation
\documentclass[12pt,a4paper,twoside,openright]{report}
\usepackage[pdfborder={0 0 0}]{hyperref}    % turns references into hyperlinks
\usepackage[margin=25mm]{geometry}  % adjusts page layout
\usepackage{graphicx}  % allows inclusion of PDF, PNG and JPG images
\usepackage{verbatim}
\usepackage{docmute}   % only needed to allow inclusion of proposal.tex


\usepackage{listings}       % Source code listings
\usepackage{courier}        % courier font
\lstset{
  basicstyle=\ttfamily
, commentstyle=\color{Green}
, keywordstyle=\bfseries\color{RoyalBlue}
, showspaces=false
, showstringspaces=false
, breaklines=true
, breakatwhitespace=true
, framextopmargin=50pt
, columns=fullflexible,keepspaces
, escapeinside={(*}{*)}
%, frame=bottomline
      }



\raggedbottom                           % try to avoid widows and orphans
\sloppy
\clubpenalty1000%
\widowpenalty1000%

\renewcommand{\baselinestretch}{1.1}    % adjust line spacing to make
                                        % more readable

\begin{document}

\bibliographystyle{plain}


%%%%%%%%%%%%%%%%%%%%%%%%%%%%%%%%%%%%%%%%%%%%%%%%%%%%%%%%%%%%%%%%%%%%%%%%
% Title


\pagestyle{empty}

\rightline{\LARGE \textbf{Cheng Sun}}

\vspace*{60mm}
\begin{center}
\Huge
\textbf{How to write a dissertation in \LaTeX} \\[5mm]
Computer Science Tripos -- Part II \\[5mm]
Churchill College \\[5mm]
\today  % today's date
\end{center}

%%%%%%%%%%%%%%%%%%%%%%%%%%%%%%%%%%%%%%%%%%%%%%%%%%%%%%%%%%%%%%%%%%%%%%%%%%%%%%
% Proforma, table of contents and list of figures

\pagestyle{plain}

\chapter*{Proforma}

{\large
\begin{tabular}{ll}
Name:               & \bf Cheng Sun                       \\
College:            & \bf Churchill College                     \\
Project Title:      & \bf An observable OCaml via C and liballocs \\
Examination:        & \bf Computer Science Tripos -- Part II, July 2017  \\
Word Count:         & \bf 1587\footnotemark[1] \\
Project Originator: & Stephen Kell                    \\
Supervisor:         & Stephen Kell                    \\
\end{tabular}
}
\footnotetext[1]{This word count was computed
by \texttt{detex diss.tex | tr -cd '0-9A-Za-z $\tt\backslash$n' | wc -w}
}
\stepcounter{footnote}


\section*{Original Aims of the Project}

%TODO
%To write a demonstration dissertation\footnote{A normal footnote without the
%complication of being in a table.} using \LaTeX\ to save
%student's time when writing their own dissertations. The dissertation
%should illustrate how to use the more common \LaTeX\ constructs. It
%should include pictures and diagrams to show how these can be
%incorporated into the dissertation.  It should contain the entire
%\LaTeX\ source of the dissertation and the makefile.  It should
%explain how to construct an MSDOS disk of the dissertation in
%Postscript format that can be used by the book shop for printing, and,
%finally, it should have the prescribed layout and format of a diploma
%dissertation.


\section*{Work Completed}

%TODO
%All that has been completed appears in this dissertation.

\section*{Special Difficulties}

%TODO
%Learning how to incorporate encapulated postscript into a \LaTeX\
%document on both Ubuntu Linux and OS X.
 
\newpage
\section*{Declaration}

I, Cheng Sun of Churchill College, being a candidate for Part II of the Computer
Science Tripos, hereby declare
that this dissertation and the work described in it are my own work,
unaided except as may be specified below, and that the dissertation
does not contain material that has already been used to any substantial
extent for a comparable purpose.

\bigskip
\leftline{Signed [signature]}

\medskip
\leftline{Date [date]}

\tableofcontents

\listoffigures

\newpage
\section*{Acknowledgements}

%TODO
%This document owes much to an earlier version written by Simon Moore
%\cite{Moore95}.  His help, encouragement and advice was greatly 
%appreciated.

%%%%%%%%%%%%%%%%%%%%%%%%%%%%%%%%%%%%%%%%%%%%%%%%%%%%%%%%%%%%%%%%%%%%%%%
% now for the chapters

\pagestyle{headings}

\chapter{Introduction}

\section{Overview of the files}

This document consists of the following files:

\begin{itemize}
\item \texttt{makefile} --- The makefile for the dissertation and
                         Project Proposal
\item \texttt{diss.tex} --- The dissertation
\item \texttt{proposal.tex}  --- The project proposal 
\item \texttt{figs} -- A directory containing diagrams and pictures
\item \texttt{refs.bib} --- The bibliography database
\end{itemize}

\section{Building the document}

This document was produced using \LaTeXe which is based upon
\LaTeX\cite{Lamport86}.  To build the document you first need to
generate \texttt{diss.aux} which, amongst other things, contains the
references used.  This if done by executing the command:

\texttt{pdflatex diss}

\noindent
Then the bibliography can be generated from \texttt{refs.bib} using:

\texttt{bibtex diss}

\noindent
Finally, to ensure all the page numbering is correct run \texttt{pdflatex}
on \texttt{diss.tex} until the \texttt{.aux} files do not change.  This
usually takes 2 more runs.

\subsection{The makefile}

To simplify the calls to \texttt{pdflatex} and \texttt{bibtex}, 
a makefile has been provided, see Appendix~\ref{makefile}. 
It provides the following facilities:

\begin{description}

\item\texttt{make} \\
 Display help information.

\item\texttt{make proposal.pdf} \\
 Format the proposal document as a PDF.

\item\texttt{make view-proposal} \\
 Run \texttt{make proposal.pdf} and then display it with a Linux PDF viewer
 (preferably ``okular'', if that is not available fall back to ``evince'').

\item\texttt{make diss.pdf} \\
 Format the dissertation document as a PDF.

\item\texttt{make count} \\
Display an estimate of the word count.

\item\texttt{make all} \\
Construct \texttt{proposal.pdf} and \texttt{diss.pdf}.

\item\texttt{make pub} \\ Make \texttt{diss.pdf}
and place it in my \texttt{public\_html} directory.

\item\texttt{make clean} \\ Delete all intermediate files except the
source files and the resulting PDFs. All these deleted files can
be reconstructed by typing \texttt{make all}.

\end{description}


\section{Counting words}

An approximate word count of the body of the dissertation may be
obtained using:

\texttt{wc diss.tex}

\noindent
Alternatively, try something like:

\verb/detex diss.tex | tr -cd '0-9A-Z a-z\n' | wc -w/


\chapter{Preparation}

%TODO


\chapter{Implementation}

%TODO

\chapter{Evaluation}

%TODO

\chapter{Conclusion}

%TODO


%%%%%%%%%%%%%%%%%%%%%%%%%%%%%%%%%%%%%%%%%%%%%%%%%%%%%%%%%%%%%%%%%%%%%
% the bibliography
\addcontentsline{toc}{chapter}{Bibliography}
\bibliography{refs}

%%%%%%%%%%%%%%%%%%%%%%%%%%%%%%%%%%%%%%%%%%%%%%%%%%%%%%%%%%%%%%%%%%%%%
% the appendices
\appendix

\chapter{Latex source}

\section{diss.tex}
{\scriptsize\verbatiminput{diss.tex}}

\section{proposal.tex}
{\scriptsize\verbatiminput{proposal.tex}}

\chapter{Makefile}

\section{makefile}\label{makefile}
%{\scriptsize\verbatiminput{makefile.txt}}

%\section{refs.bib}
%{\scriptsize\verbatiminput{refs.bib}}


\chapter{Project Proposal}

%\documentclass[12pt,twoside,a4paper]{article}

\usepackage[pdfborder={0 0 0}]{hyperref}
\usepackage[margin=25mm]{geometry}
\usepackage{parskip}

\usepackage{a4}             % Adjust margins for A4 media

\usepackage{lastpage}       % "n of m" page numbering
\usepackage{lscape}         % Makes landscape easier
%\usepackage{portland}       % Switch between portrait and landscape
\usepackage[usenames,dvipsnames,svgnames,table]{xcolor}         % X11 colour names
\usepackage{graphics}       % Graphics commands
\usepackage{wrapfig}        % Wrapping text around figures
\usepackage{epsfig}         % Embed encapsulated postscript
\usepackage{rotating}       % Extra graphics rotation
%\usepackage{tables}         % Tabular environments
\usepackage{longtable}      % Page breaks within tables
\usepackage{supertabular}   % Page breaks within tables
\usepackage{multicol}       % Allows table cells to span cols
\usepackage{multirow}       % Allows table cells to span rows
%\usepackage{texnames}       % Macros for common tex names
%\usepackage{trees}          % Tree-like layout
\usepackage{mdframed}       % frames around paragraphs

\usepackage{listings}       % Source code listings
\usepackage{courier}        % courier font
\lstset{
  basicstyle=\ttfamily
, commentstyle=\color{Green}
, keywordstyle=\bfseries\color{RoyalBlue}
, showspaces=false
, showstringspaces=false
, breaklines=true
, breakatwhitespace=true
, framextopmargin=50pt
, columns=fullflexible,keepspaces
, escapeinside={(*}{*)}
%, frame=bottomline
      }
\usepackage{array}          % Array environment
\usepackage[shortlabels]{enumitem}       % fancy enum settings
\usepackage{url}            % URL formatting
\usepackage{amsmath}        % American Mathematical Society
\usepackage{amssymb}        % Maths symbols
\usepackage{amsthm}         % Theorems
%\usepackage{mathpartir}     % Proofs and inference rules
\usepackage{verbatim}       % Verbatim blocks
\usepackage{fancyvrb}       % Verbatim blocks
\usepackage{ifthen}         % Conditional processing in tex
\usepackage{caption}         % \caption*, no colon

\usepackage{graphicx}
\usepackage{tikz}
\usepackage[export]{adjustbox} %valign=t
\usepackage{siunitx}        % non-slanted units in math mode with \SI{1}{\volt}
\usepackage{nicefrac}          % slanted fractions with \nicefrac{1}{2}

\newcommand*\BitAnd{\mathrel{\&}}
\newcommand*\BitOr{\mathrel{|}}
\newcommand*\ShiftLeft{\ll}
\newcommand*\ShiftRight{\gg}
\newcommand*\BitNeg{\ensuremath{\mathord{\sim}}}
\usepackage{mathtools}
\DeclarePairedDelimiter{\ceil}{\lceil}{\rceil}
\mathtoolsset{showonlyrefs=true} % show only eq refs that are used

\usepackage{float}

\renewcommand{\thispagestyle}[1]{}

\setlength{\parindent}{0em}

\begin{document}

\begin{center}
\Large Computer Science Tripos -- Part II -- Project Proposal
\\[4mm]
\LARGE \bfseries An observable OCaml, via C and \lstinline{liballocs} \mdseries
\\[4mm]

\large
Cheng Sun, Churchill College

Originator: Stephen Kell

\makeatletter
\@date
\makeatother

\end{center}

\vspace{5mm}

\textbf{Project Supervisor:} Stephen Kell

\textbf{Director of Studies:} John Fawcett

\textbf{Project Overseers:} Timothy Griffin \& Pietro Lio


\section*{Introduction}

OCaml is one of the most commonly used members of the ML family of functional languages. It is popular for its expressivity, type system and performance. However, there is not as of yet a good story for debugging OCaml programs, and observing their behaviour at runtime.

The OCaml bytecode debugger, \lstinline{ocamldebug}, forms part of the core OCaml toolchain. One major problem is that \lstinline{ocamldebug} is unable to ``see through'' polymorphism. For instance, suppose we would like to debug the following polymorphic list-reverse function, which has type \lstinline{'a list -> 'a list}:

\begin{lstlisting}
let my_rev lst =
  match lst with
  | [] -> []
  | x::xs -> List.append xs [x]

let result = my_rev [1; 2; 3]
\end{lstlisting}

Now let's try to debug the function with \lstinline{ocamldebug}.

\begin{lstlisting}
$ (!\aftergroup\bfseries!)ocamlc -g -o my_rev my_rev.ml(!\aftergroup\mdseries*)
$ (!\aftergroup\bfseries!)ocamldebug my_rev(!\aftergroup\mdseries!)
        OCaml Debugger version 4.02.3

(ocd) (!\aftergroup\bfseries!)break @ My_rev 1(!\aftergroup\mdseries!)
Loading program... done.
Breakpoint 1 at 21600: file my_rev.ml, line 2, characters 3-62
(ocd) (!\aftergroup\bfseries!)run(!\aftergroup\mdseries!)
Time: 12 - pc: 21600 - module My_rev
Breakpoint: 1
2   <|b|>match lst with
(ocd) (!\aftergroup\bfseries!)print lst(!\aftergroup\mdseries!)
lst: 'a list = [<poly>; <poly>; <poly>]
\end{lstlisting}

Note that \lstinline{ocamldebug} is unable to display the contents of the input list, as it does not know the concrete type that the type variable \lstinline{'a} is instantiated with for this invocation.

There are many other deficiencies with the OCaml debugger, mostly stemming from its immaturity and lack of features. These issues mean that when debugging OCaml code, one often has to resort to ``printf debugging'' instead.

The aim of this project is to investigate whether the experience of debugging and observing the runtime behaviour of OCaml programs could be improved by utilising the mature C toolchain. My goal is to write a translator that compiles OCaml code into equivalent C code, whilst maintaining a well-defined mapping between the two (in terms of variable names, types, and so on). A user will then be able to debug their OCaml program by using (perhaps an augmented) \lstinline{gdb} on the generated C code.

In order to solve the problem of ``seeing through'' polymorphism, I will use a library written by my supervisor, \lstinline{liballocs}, which keeps track of runtime allocation metadata (including their types) with low overhead. This will allow the debugger to inspect the allocation corresponding to the input to \lstinline{my_rev}, for instance, and conclude that it is operating on lists with elements of type \lstinline{int}.


\section*{Starting point}

The project will be focused on creating a new ``backend'' for the OCaml
compiler, so I will build on the existing code of the frontend.
This includes reusing code for lexing, parsing and typing of OCaml programs, as
well as the transformation from the typed AST to the Intermediate
Representation that we will be using as input.

In order to provide a standard library to compile programs against, I will try
to use as much of the existing OCaml \lstinline{stdlib} as possible. If this
turns out not to be feasible then I will resort to writing a subset of the
standard library by hand.

I will use the open-source library \lstinline{liballocs}, written by Stephen Kell.
This library provides routines to tag regions in virtual memory
(``allocations'') with run-time type information.

I may also make use of some further open-source libraries such as:
\begin{itemize}
  \item \lstinline{libffi} -- a library providing a portable way to perform a call to a function
with a foreign function interface (such as the closures that I will dynamically
create at runtime);
  \item \lstinline{cil} -- a library written in OCaml providing a framework for the manipulation of C programs.
\end{itemize}

There exists similar prior work \cite{tarditi90} that shows that the overall project concept is feasible. However, I will be making different design choices to the compiler presented in the paper, due to
differing requirements. For instance, the presented compiler
generates code that uses continuation-passing style, which I would like to
avoid doing, as it harms observability -- backtraces would no longer be directly
meaningful.


\section*{Resources required}

The project will primarily be developed on my personal computer
for convenience. The computer has a 4-core 3.30GHz Intel Xeon
CPU, a 250GB SSD and 8GB of RAM. I am using Arch Linux. No other non-standard equipment
is anticipated to be required, and I can easily continue development on
MCS machines if required.

The project will be synced to a git server (likely GitHub), and backed up
regularly to my MCS network drive.

The project will utilise various standard open-source software packages,
including the usual C toolchain (e.g. \lstinline{gcc}, \lstinline{gdb}, \lstinline{make}) along with
the OCaml compiler. (As mentioned previously, \lstinline{liballocs} requires
some patches to \lstinline{binutils}, but these can easily be built.)

\section*{Work to be done}

The project can be split up into the following tasks:

\begin{enumerate}
  \item
    Performing a study of the starting point (OCaml compiler frontend and \lstinline{liballocs}).
    This includes an investigation into the forms of the various Intermediate Representations of the OCaml compiler, and selection of the best IR to use as input for our project.
  \item
    Choosing a suitable object representation for boxed objects in memory. There are a variety of alternatives to OCaml's own tagged-pointer representation (which leads to awkward 63-bit integers).
    This choice requires consideration of the garbage collection strategy, even if a GC is not within the core scope of this project.
  \item
    Implementing the translation from OCaml IR to C (with \lstinline{liballocs}), for a variety of language features:
    \begin{enumerate}
      \item Fundamental types: \lstinline{bool}, \lstinline{int}, \lstinline{float}, \lstinline{list}, \lstinline{ref}, \ldots;
      \item Tuples and records, which can be implemented using C \lstinline{struct}s;
      \item (Non-polymorphic) variants, which can be implemented using tagged C \lstinline{union}s;
      \item Polymorphic types, which can be represented safely using \lstinline{void *}
        opaque pointers, as downcasts are never required
        in ML (although \lstinline{liballocs} could provide the necessary information);
      \item Parametrically polymorphic functions;
      \item First class functions and closures, which require lambda lifting,
        or more generally something like the technique of Breuel
        \cite{breuel88} (dynamically creating an executable stub associated
        with each closure at runtime).
    \end{enumerate}
  \item
    Integrating a subset of the OCaml \lstinline{stdlib}, or writing my own as necessary.
  \item
    Writing a runtime library to support dynamic closure creation.
  \item
    Creating a corpus of test and benchmark programs.
  \item
    Evaluation, as described in the \textit{Success criteria} section below.
  \item
    Extensions as time permits, as described in the \textit{Possible extensions} section below.
\end{enumerate}

\section*{Success criteria}

My project will have been successful if I have managed to create a OCaml-to-C translator that works on a commonly used subset of the OCaml language (including closures, parametrically polymorphic functions). Furthermore the polymorphism must be able to be seen through (solving the problem highlighted in the \textit{Introduction} section, for instance).

The project will be evaluated in two ways. Both will require a corpus of OCaml tests and micro-benchmarks.

Firstly, I will measure the ``observability'' of the compiled C programs compared with their native and bytecode OCaml counterparts. For instance, one metric would be obtained by interrupting these programs at pre-determined points in the code, and counting the number of local variables that could be recovered from the stack frames. This is an accurate metric for observability, as the utility of debugging is determined almost completely by the amount of state that is visible at a breakpoint.

Secondly, I will measure the performance of the resultant C executables, by timing benchmark programs as compiled by both compilers. As the primary objective of the project is debuggability rather than performance, I do not expect the performance to be anywhere near as good as code generated by the OCaml native-code compiler: I will be satisfied as long as the performance is within a reasonable factor for the majority of use cases. It is still necessary to evaluate performance, because it affects the debuggability of CPU-intensive programs.

There are many other qualitative evaluations that I will also consider, such as the ease of interoperability with C libraries using this approach, the ability to insert instrumentation into the code, and the overall debugging experience when using \lstinline{gdb} on the resultant executables.

\section*{Possible extensions}

\begin{itemize}
  \item Improving the runtime, such as adding a garbage collector;
  \item Adding further language and standard library support, to increase the range of programs supported;
  \item Optimising the performance of the generated program;
  \item Improving the debugging experience (such as augmenting \lstinline{gdb}).
\end{itemize}

\section*{Timetable}

\begin{enumerate}
\item \textbf{Michaelmas weeks 2--3:}
  Dig into the OCaml compiler frontend; study its Intermediate Representations. Learn to use the \lstinline{liballocs} library.

\item \textbf{Michaelmas weeks 4--6:}
  Begin work on translator, initially targeting basic features (fundamental types, monomorphic functions).
  Design an initial representation for boxed objects.

\item \textbf{Michaelmas weeks 7--8:}
  Add support for tuples, variants, records. Test against very simple programs.

\item \textbf{Michaelmas vacation weeks 1--2:}
  Add support for polymorphic functions and types.

\item \textbf{Michaelmas vacation weeks 3--4:}
  Get a small, commonly-used subset of the standard library working.
  Start creating corpus of test programs, and begin testing.

\item \textbf{Michaelmas vacation weeks 5--6:}
  Reserved for holidays/revision.

\item \textbf{Lent term weeks 0--1:}
  Add initial support for dynamic closure creation.
  Complete corpus of test OCaml programs; begin evaluation.
  Start progress report.

\item \textbf{Lent term week 2--3:}
  Milestone: compiler has working support for most test programs (demoable).
  Submit progress report.
  Prepare and give presentation.

\item \textbf{Lent term weeks 4--6:}
  Testing; fix bugs. Improve closure creation, and evaluate.
  Investigate ``source maps'' to map C code to corresponding lines in the OCaml source.

\item \textbf{Lent term weeks 7--8:}
  Further evaluation, and work on improving evaluated metrics.
  Start on main chapters of dissertation.

\item \textbf{Easter vacation weeks 1--3:}
  Milestone: compiler works for all intended features.
  Wrap up evaluation. Continued work on dissertation.

\item \textbf{Easter vacation weeks 4--5:}
  Reserved for holidays/revision. Extension work if time permits.

\item \textbf{Easter term weeks 0--2:}
  Form conclusion and complete dissertation. Time may be limited due to revision.

\item \textbf{Easter term week 3:}
  Proof reading and submission of final dissertation.



\end{enumerate}

\begin{thebibliography}{9}

\bibitem{tarditi90}
  D. Tarditi, A. Acharya, P. Lee,
  \emph{No Assembly Required: Compiling Standard ML to C},
  November 1990. \\ \url{http://repository.cmu.edu/cgi/viewcontent.cgi?article=3011&context=compsci}

\bibitem{breuel88}
  Thomas Breuel,
  \emph{Lexical Closures for C++},
  In Proc. USENIX C++ Conf., pages 293-304,
  Denver, CO, October 1988.  \\ \url{http://www.cl.cam.ac.uk/~srk31/teaching/redist/breuel88lexical.pdf}

\end{thebibliography}

\end{document}


\end{document}
