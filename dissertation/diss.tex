% Template for a Computer Science Tripos Part II project dissertation
\documentclass[12pt,a4paper,twoside,openright]{report}
\usepackage[pdfborder={0 0 0}]{hyperref}    % turns references into hyperlinks
\usepackage[margin=25mm]{geometry}  % adjusts page layout
\usepackage{graphicx}  % allows inclusion of PDF, PNG and JPG images
\usepackage{verbatim}
\usepackage{docmute}   % only needed to allow inclusion of proposal.tex
\usepackage{amsmath}


\usepackage{listings}       % Source code listings
\usepackage{courier}        % courier font
\usepackage{textcomp}       % straight quotes
\lstset{
  basicstyle=\ttfamily
, commentstyle=\color{Green}
, keywordstyle=\bfseries\color{RoyalBlue}
, showspaces=false
, showstringspaces=false
, breaklines=true
, breakatwhitespace=true
, framextopmargin=50pt
, columns=fullflexible,keepspaces
, escapeinside={(*}{*)}
, upquote=true
%, frame=bottomline
      }



\raggedbottom                           % try to avoid widows and orphans
\sloppy
\clubpenalty1000%
\widowpenalty1000%

\renewcommand{\baselinestretch}{1.1}    % adjust line spacing to make
                                        % more readable

\begin{document}

\bibliographystyle{plain}


%%%%%%%%%%%%%%%%%%%%%%%%%%%%%%%%%%%%%%%%%%%%%%%%%%%%%%%%%%%%%%%%%%%%%%%%
% Title


\pagestyle{empty}

\rightline{\LARGE \textbf{Cheng Sun}}

\vspace*{60mm}
\begin{center}
\Huge
\textbf{An observable OCaml via C and \texttt{liballocs}} \\[5mm]
Computer Science Tripos -- Part II \\[5mm]
Churchill College \\[5mm]
\today  % today's date
\end{center}

%%%%%%%%%%%%%%%%%%%%%%%%%%%%%%%%%%%%%%%%%%%%%%%%%%%%%%%%%%%%%%%%%%%%%%%%%%%%%%
% Proforma, table of contents and list of figures

\pagestyle{plain}

\chapter*{Proforma}

{\large
\begin{tabular}{ll}
Name:               & \bf Cheng Sun                       \\
College:            & \bf Churchill College                     \\
Project Title:      & \bf An observable OCaml via C and \texttt{liballocs} \\
Examination:        & \bf Computer Science Tripos -- Part II, July 2017  \\
Word Count:         & \bf 1587\footnotemark[1] \\
Project Originator: & Stephen Kell                    \\
Supervisor:         & Stephen Kell                    \\
\end{tabular}
}
\footnotetext[1]{This word count was computed
by \lstinline!detex diss.tex | tr -cd '0-9A-Za-z \\n' | wc -w!
}
\stepcounter{footnote}


\section*{Original Aims of the Project}

%TODO
%To write a demonstration dissertation\footnote{A normal footnote without the
%complication of being in a table.} using \LaTeX\ to save
%student's time when writing their own dissertations. The dissertation
%should illustrate how to use the more common \LaTeX\ constructs. It
%should include pictures and diagrams to show how these can be
%incorporated into the dissertation.  It should contain the entire
%\LaTeX\ source of the dissertation and the makefile.  It should
%explain how to construct an MSDOS disk of the dissertation in
%Postscript format that can be used by the book shop for printing, and,
%finally, it should have the prescribed layout and format of a diploma
%dissertation.


\section*{Work Completed}

%TODO
%All that has been completed appears in this dissertation.

\section*{Special Difficulties}

%TODO
%Learning how to incorporate encapulated postscript into a \LaTeX\
%document on both Ubuntu Linux and OS X.
 
\newpage
\section*{Declaration}

I, Cheng Sun of Churchill College, being a candidate for Part II of the Computer
Science Tripos, hereby declare
that this dissertation and the work described in it are my own work,
unaided except as may be specified below, and that the dissertation
does not contain material that has already been used to any substantial
extent for a comparable purpose.

\bigskip
\leftline{Signed [signature]}

\medskip
\leftline{Date [date]}

\tableofcontents

\listoffigures

\newpage
\section*{Acknowledgements}

%TODO
%This document owes much to an earlier version written by Simon Moore
%\cite{Moore95}.  His help, encouragement and advice was greatly 
%appreciated.

%%%%%%%%%%%%%%%%%%%%%%%%%%%%%%%%%%%%%%%%%%%%%%%%%%%%%%%%%%%%%%%%%%%%%%%
% now for the chapters

\pagestyle{headings}

\chapter{Introduction}

%TODO


\chapter{Preparation}

\section{The OCaml compiler}

The vast majority of my code will be written as part of the OCaml compiler. The compiler is itself written in OCaml, and is a large and complex code base developed over the course of over 20 years, consisting of 197k lines of OCaml and 42k lines of C.\footnote{As counted by the \lstinline!cloc! tool.}

The OCaml compiler processes OCaml code in several phases:

TODO diagram

\subsection{Common types}\label{common-types}

The compiler represents the name of all variables as \lstinline!Ident.t!, which
is conceptually a tuple of the variable name string and a unique integer. Each
let binding is given its own unique \lstinline!Ident.t! -- once the frontend
has resolved variables using OCaml's scoping and shadowing rules, the rest of
the compiler has the guarantee of globally unique\footnote{Up to module
boundary -- at the \lstinline!Lambda! level intermodule variable references
have already boiled down to record field accesses.} identifier names.

A \lstinline!type_expr! represents the OCaml type of a particular expression.

\subsection{\texttt{Parsetree} and \texttt{Typedtree}}

The very first representation used by the compiler is \lstinline!Parsetree!.
This is an Abstract Syntax Tree produced directly from the OCaml source code by
the parser. At this stage all original OCaml features are still present, before
any desugaring has happened yet. For instance, pattern matches are as
originally expressed in the source code.

The typing phase takes a \lstinline!Parsetree! and produces a
\lstinline!Typedtree!. This is the phase that ensures that all expressions are
well-typed. The \lstinline!Typedtree! is almost identical to
\lstinline!Parsetree!. The difference is that every \lstinline!expression! node
in the tree is augmented with the \lstinline!type_expr! -- the expression of
this type.

\subsection{\texttt{Lambda} IR}

For our purposes, \lstinline!Lambda! is the most important representation in
the OCaml compiler, as it is the starting point for our compiler. This
Intermediate Representation (IR) is based on a heavily augmented lambda calculus.

The pass that compiles \lstinline!Typedtree! into \lstinline!Lambda! IR most
significantly desugars pattern matching into elementary conditionals and
destructuring operations. Also class and module constructs are compiled down to
records.

Notably this pass does NOT provide:

\begin{itemize}
    \item type preservation -- see section \ref{type-information-for-liballocs}
    \item closure conversion -- see section \ref{closures}
    \item uncurrying\footnote{Significantly for us this means
        partial applications and ``overapplications'' are not converted into
        complete applications} -- see section \ref{closures-compiler-support}
\end{itemize}


\section{\texttt{liballocs}}

\lstinline!liballocs! is TODO


\chapter{Implementation}

\section{Summary and structure of code}

\begin{enumerate}
  \item the driver
  \item the compilation module
    \begin{enumerate}
      \item AST
      \item translate
      \item fixup
      \item emitcode
    \end{enumerate}
  \item the runtime
    \begin{enumerate}
      \item OCaml standard library
      \item C runtime library
    \end{enumerate}
\end{enumerate}

The stages of my C compiler backend are as follows. TODO

\subsection{\texttt{module C}: the extended C abstract syntax tree datatype}

I have defined a datatype representing a C-like syntax tree. Not all C features
are modelled -- only the constructs that my compiler needs to produce.

The tree datatype is slightly more expressive than standard ANSI C. These extra
features are rewritten into standard C in a later pass, as detailed later in
section \ref{module-fixup}.

The following datatypes are defined:

\begin{itemize}
  \item
    \lstinline!C.ctype! represents a C type. Notable constructors include:

    \begin{itemize}
        \item \lstinline!C_Pointer of C.ctype!
        \item \lstinline!C_Boxed! representing a generic OCaml value -- either
            an immediate int, immediate double or a pointer to a heap-allocated
            object. The representation of this is discussed in detail in
            section \ref{nan-boxing}.
        \item \lstinline!C_Int!
        \item \lstinline!C_Double!
        \item \lstinline!C_Struct of Ident.t * (C.ctype * string) list!
        \item \lstinline!C_FunPointer of C.ctype * C.ctype list! taking return
            type and argument types
    \end{itemize}

    For instance, a function pointer of type \lstinline!void (*)(double *)! is
    represented as \lstinline!C_FunPointer (C_Void, C_Pointer C_Double)!.
  \item
    \lstinline!C.expression! represents an expression -- program fragments that
    evaluate to a value (possibly of type \lstinline!void!). Notable
    constructors for this datatype include:

    \begin{itemize}
        \item Data literals of various types:
            \begin{itemize}
                \item \lstinline!C_IntLiteral of int64!
                \item \lstinline!C_StringLiteral of string!
                \item \lstinline!C_FloatLiteral of string!\footnote{Floats are
                    always represented in the OCaml compiler as strings, to
                    prevent rounding errors.}
                \item \lstinline!C_CharLiteral of char!
                \item \lstinline!C_PointerLiteral of int!\footnote{This
                    doesn't take \lstinline!int64! as might be expected, as it
                    is used to represent lambda's \lstinline!Const_pointer of int!,
                    which seems only to be used by the compiler to represent
                    \lstinline!NULL!.}
            \end{itemize}
        \item Variables. Two different kinds are distinguished:
          \begin{itemize}
            \item \lstinline!C_Variable of Ident.t! takes an identifier (see section
              \ref{common-types}), and is used for all local variables
              originating from OCaml code. The variables are formatted suffixed
              with their unique integer: \lstinline!variablename_1234!.
            \item \lstinline!C_GlobalVariable of string! takes a string and
              outputs it verbatim -- there is no integer suffix. The compiler
              generates these references when it needs to reference internal
              functions, such as \lstinline!ocaml_liballocs_close! (see section
              \ref{closures-runtime-support}). Cross-references to other
              modules also use these.
          \end{itemize}
        \item \lstinline!C_FunCall of C.expression * C.expression list!
            represents function calls as an expression evaluating to the
            function to be called, and the list of arguments. See section \ref{functions}.
        \item \lstinline!C_BinaryOp of string * expression * expression!
            represents all arithmetic and logical binary operations.
    \end{itemize}

    For instance, the expression \lstinline!foo + 2! is represented as
    \begin{lstlisting}
    C_BinaryOp (C_Variable foo, "+", C_IntLiteral (Int64.of_int 2))\end{lstlisting}%
    (where \lstinline!foo : Ident.t! is the identifier representing the name of the
    \lstinline!foo! variable).

	\item
		\lstinline!C.statement! represents a statement -- language constructs such as:

        \begin{itemize}
            \item \lstinline!C_If of expression * statement list * statement list!
            \item \lstinline!C_While of expression * statement list!
            \item \lstinline!C_Return of expression option!
            \item \lstinline!C_Expression of expression!
            \item etc.
        \end{itemize}
        Note that statements typically contain one or more expressions, whereas
        expressions in standard C cannot contain statements. (This is however
        possible in my AST -- as described below in sections \ref{c-inline-statements} and \ref{c-inline-functions}.)

	\item
		\lstinline!C.toplevel! represents a function or global variable declaration/definition at the toplevel.
        TODO
\end{itemize}

Refer to (appendix -- section TODO) for the complete datatype definition.

\subsection{\texttt{module Translate}: compilation from lambda IR to C AST}

The implementation of \lstinline!Translate! is done via a recursive tree-walk
over the \lstinline!Lambda! tree. Three mutually recursive functions each
target different types of C program fragment:

TODO

Unfortunately the lambda IR is not documented anywhere, so the semantics
of these instructions were divined through reverse engineering and code diving.
OCaml's expressive type system often hinted at the semantics of various
parts of the code.

\subsubsection{Primitive operations}

\begin{enumerate}
  \item \lstinline!Pidentity! TODO
  \item \lstinline!Pignore! TODO
  \item \lstinline!Popaque! TODO
  \item \lstinline!Pdirapply! and \lstinline!Prevapply! TODO
  \item \lstinline!Pgetglobal! TODO
  \item \lstinline!Pfield! and \lstinline!Pfloatfield! TODO
  \item \lstinline!Psetfield! TODO
  \item \lstinline!Pmakeblock! TODO
  \item various unary and binary operations TODO
  \item \lstinline!Pccall! TODO
  \item \lstinline!Praise! TODO
  \item \lstinline!Pstringlength! TODO
  \item \lstinline!Pstringrefs! TODO
\end{enumerate}

\subsubsection{Structured constants}

TODO

\subsubsection{Events}

\subsubsection{Other \texttt{Lambda} operations}

\begin{enumerate}
  \item \lstinline!Lvar! TODO
  \item \lstinline!Lfunction! TODO
  \item \lstinline!Lapply! TODO

    ---

  \item \lstinline!Llet!, \lstinline!Lletrec! TODO
  \item \lstinline!Lsequence! TODO
  \item \lstinline!Lifthenelse! TODO
  \item \lstinline!Lwhile! -- see section \ref{while-loops}
  \item \lstinline!Lfor! -- see section \ref{for-loops}
  \item \lstinline!Lstringswitch! -- see section \ref{stringswitch}
  \item \lstinline!Ltrywith!, \lstinline!Lstaticcatch!, \lstinline!Lstaticraise! -- see section \ref{exceptions}
\end{enumerate}

\subsubsection{Inline statements}\label{c-inline-statements}

The fact that C language makes a distinction between ``statements'' and
``expressions'' makes it less expressive than OCaml, as all core language
features in OCaml can be used as expressions. For instance, consider the
translation of the following OCaml code:

\begin{lstlisting}
let x = if foo then (if bar then 1 else 2) else 3
\end{lstlisting}

Clearly a direct translation of this wouldn't be possible, as \lstinline!if!
statements cannot be used as expressions in C. The ternary operator
\lstinline!?:! exists in C as an expression analogue of \lstinline!if!.
However, it isn't used by my compiler for two reasons.  Firstly in the
interests of readability -- nested ternary operators get hard to read very
quickly. The second reason is that debuggability of if statements would be
hampered by the fact that debuggers tend not to be able to single-step easily
from the condition to the body.

Even if the ternary operator were used, there are other statements that
cannot be made into expressions as easily. An example is the let-statement,
which translates to variable declarations in C. Hence, a solution to this
general problem would still be required.

The way that we handle this is by extending our \lstinline!C.expression! so
that statements can become expressions.  We add a new constructor
\lstinline!C_InlineRevStatements of C.statement! to the
\lstinline!C.expression! type, which allows us to use any C statement as if it
were an expression.

By doing this, our example from above can be translated directly, simply by
wrapping the \lstinline!C_If! statements up as an ``inline statement block'',
which are a list of statements disguised as a single expression, where an
inline statement evaluates to the ``value'' of the last statement in the block.

See section \ref{module-fixup} for more details.


\subsubsection{Inline function definitions (lambdas)}\label{c-inline-functions}

\begin{lstlisting}
let x = (fun x -> x + 1) 2
\end{lstlisting}

\subsection{\texttt{module Fixup}: translation from extended-C AST to valid C AST}\label{module-fixup}

Similar to \lstinline!Translate!, the implementation of \lstinline!Fixup! is
done via a recursive tree-walk. Three mutually recursive functions each visit
different types of nodes:

\begin{lstlisting}
var fixup_let_defs :
  Fixup.t -> C.statement list -> C.let_definition list ->
  C.statement list

var fixup_expression :
  Fixup.t -> C.statement list -> C.expression ->
  C.statement list * C.expression

var fixup_rev_statements :
  Fixup.t -> C.statement list -> C.statement list ->
  C.statement list
\end{lstlisting}

Each fixup function is invoked in the form \lstinline!fixup_foo t accum foo!

We define a single operation which is used during the process of extracting an
expression from an inline block:

\begin{lstlisting}
assign_last_value_of_statement
\end{lstlisting}

TODO

\subsection{\texttt{module Emitcode}: outputting C AST to a file}\label{emitcode}

TODO

Not all OCaml identifiers are legal C identifiers. TODO

\subsection{Driver modification}

Support was added to the \lstinline!ocamlc! driver binary to generate and
output C code. A new flag \lstinline!-target-liballocs! was added. After the
simplified lambda IR is generated as usual, the presence of this flag will
trigger the additional phase of compiling and outputting the C code in addition
to running the rest of the compiler pipeline.

\section{Build and test scripts}

TODO


\section{Basic constructs}

\subsection{Functions and complete application}\label{functions}

TODO

Partial application and closures are discussed later in section \ref{closures}.

\subsection{Mutually recursive functions}\label{mutually-recursive-functions}

TODO

\subsection{Mutually recursive definitions}\label{mutually-recursive-values}

TODO

Consider the OCaml expression

\begin{lstlisting}
let rec a = 1::b and b = 2::a in ...
\end{lstlisting}

This defines a mutually recursive \textit{value} -- creating a infinite
(cyclic) list of period 2. The value of \lstinline!a! will be
\lstinline!1::2::1::2::1:: ...!

In order to ensure correct semantics even in the presence of mutually recursive
value definitions, my compiler separates the variable definition process into
three phases:

\begin{enumerate}
	\item \textbf{Allocation}: first the required memory for each variable are
		allocated using \lstinline!malloc!;
	\item \textbf{Closure creation (optional)}: (TODO: move this to the closure section)  at this point the memory
		addresses of variables are known, and so can be frozen into closures as
		needed. The process is described in great detail in section \ref{closures}
		(TODO: actually write this section);
	\item \textbf{Initialisation}: finally, the contents of the variables are
		set.
\end{enumerate}

The resulting C code will look like:

\begin{lstlisting}
// Phase 1: allocation
ocaml_value_t a_1213 = NEW_P(malloc(sizeof(ocaml_value_t) * 2));
ocaml_value_t b_1213 = NEW_P(malloc(sizeof(ocaml_value_t) * 2));

// Phase 3: initialisation
a_1213[0] = 1;
a_1213[1] = b_1213;
b_1213[0] = 2;
b_1213[1] = a_1213;
\end{lstlisting}
(TODO: fix this once list malloc typing works!)

\subsection{Imperative-style while loops}\label{while-loops}

TODO

\subsection{Range-based for loops}\label{for-loops}

OCaml's for loops are range-based, coming in two varieties:
\begin{lstlisting}
for i = 1 to 10 do ... done
for i = 10 downto 1 do ... done
\end{lstlisting}

The range limits are inclusive.

This needs to be translated at some point in into C's more general
initialisation-condition-update for loop. I chose to reflect OCaml's for loop
semantics rather than C's with my \lstinline!C.expression!
constructor, deferring the translation to the \lstinline!Emitcode! stage (see
section \ref{emitcode}).

\lstinline!C_ForInt!

For instance, the first loop from the example above
compiles to:

\begin{lstlisting}
for (int64_t i_1203 = 1, _limit_1212 = 10;
     (i_1203<=_limit_1212);
     (++i_1203)) {
  ...
}
\end{lstlisting}

There's a slight subtlety in that we need to evaluate the range limits exactly
once, for correct semantics when evaluating the range incurs side effects.
Hence the end limit can't be evaluated in the condition part of the C for loop.
Instead we define the end limit as an additional variable in the initialisation
step, and test against the variable instead.

\subsection{String switches}\label{stringswitch}

TODO

\section{Standard library: \texttt{module Pervasives}}

\subsection{C runtime}

The Pervasives module exposes (and internally uses) a large number of external
predefined runtime functions (such as \lstinline!caml_sys_exit!), normally
provided by the OCaml C runtime library. However, this library makes many
assumptions about OCaml's tagged value representation, and is deeply integrated
with the garbage collector. Hence new implementations of these functions had to
be made.

I selectively implemented functions that were necessary for the subset of
stdlib functionality that was required to run my tests and benchmarks.

See appendix TODO for details.

\subsection{Printing to \texttt{stdout} and \texttt{stderr}}\label{pervasives-printing}

Two sets of C runtime functions had to be implemented in order to support
Pervasives functions such as \lstinline!print_int!. Int-to-string formatting
and string printing are all delegated to the runtime library.

TODO

\section{Inter-module linking}

Each \lstinline!.ml! file in OCaml implicitly defines a toplevel module. The
approach I have taken is to represent each such module as an array of values,
the order of which is defined by the OCaml compiler.

At the lambda IR all inter-modular value/function accesses have already been
lowered to \lstinline!getfield! operations on module objects. For instance, a
call to \lstinline!Foo.bar! will already be represented by something like
\lstinline!getfield Foo 1!. The ordering of values in modules is determined by
the OCaml compiler based on the module's interface (\lstinline!.mli! file if it
exists). This scheme implicitly handles data abstraction for us.

A module \lstinline!foo.ml! generates a file \lstinline!foo.c! declaring global
variable \lstinline!ocaml_value_t *Foo! (the ``module object'') and global
function \lstinline!void Foo__init()! (the ``module constructor''). This module
constructor function is idempotent, and after a call to it the module object is
guaranteed to contain valid values. Furthermore all global side effects in
\lstinline!foo.ml! are also performed when the module is initialised for the
first time.

Note that \lstinline!Foo! may well depend on other modules, such as
\lstinline!Pervasives!. In \lstinline!foo.c! any such dependencies are
represented as predeclarations of the module objects and constructors, and
calls to constructors at the start of \lstinline!Foo__init!. The skeleton of
such a module is shown in the listing below:

\begin{lstlisting}
#include "liballocs_runtime.h"

void Pervasives__init();
extern ocaml_value_t *Pervasives;

ocaml_value_t *Foo;

void Foo__init() {
    if (Foo) {
        return;
    } else {
        Pervasives__init();

        // allocate and construct module object Foo
    }
}
\end{lstlisting}

\section{Type information for \texttt{liballocs}}\label{type-information-for-liballocs}

\subsection{\texttt{Pmakeblock}}

TODO

\subsection{\texttt{structured\char`_constant}}

TODO

\section{NaN boxing}\label{nan-boxing}

In order to support polymorphic comparisons, we need to use a technique to be
able to distinguish pointers from immediate ints and doubles.

We use a technique called NaN boxing. Inspired by the WebKit JSCore

\begin{tabular}{c c | c c | c}
    63..52 & 51..0 & 63..52 & 51..0 & purpose \\
    \lstinline!fff! & \lstinline!0000000000000! & \lstinline!fff! & \lstinline!1000000000000! & Negative infinity \\
    \lstinline!fff! & \lstinline!4000000000000! & \lstinline!fff! & \lstinline!5000000000000! & Negative signalling NaN (canonical) \\
    \lstinline!fff! & \lstinline!8000000000000! & \lstinline!fff! & \lstinline!9000000000000! & Negative quiet NaN (canonical) \\
    \lstinline!fff! & \lstinline!b000000000000! & \lstinline!fff! & \lstinline!c000000000000! & Integers (50-bit space) \\
    \lstinline!fff! & \lstinline!effffffffffff! & \lstinline!fff! & \lstinline!fffffffffffff! & Integers (50-bit space) \\
    \lstinline!fff! & \lstinline!f000000000000! & \lstinline!000! & \lstinline!0000000000000! & Pointers (48-bit space) \\
    \lstinline!fff! & \lstinline!fffffffffffff! & \lstinline!000! & \lstinline!0ffffffffffff! & Pointers (48-bit space)
\end{tabular}

\subsection{50-bit integer arithmetic}\label{50-bit-integer}

NaN boxing means that our integers are 50-bit sized. This is OK, because the size of the standard unboxed integer in OCaml is deliberately left unspecified -- the upstream compiler uses either 31-bit or 63-bit integers depending on the platform.

Interestingly, using a non-word-sized integer

\section{Exceptions}\label{exceptions}

OCaml has support for exceptions: a \lstinline!raise Foo! causes the
exception \lstinline!Foo! to propagate up through stack frames until a
\lstinline!try _ with Foo -> _! block is found.

Interestingly, exceptions is one of the cases where the lambda IR is more
complex than the OCaml language. Lambda distinguishes between static (local)
exceptions, and non-static ones.

Once again, because of the undocumented nature of the lambda IR, the semantics
of these instructions were divined through reverse engineering and code diving.

\subsection{Static exceptions}

Static exceptions are confined to a lexically local scope -- i.e. the raise and
the corresponding try are in the same function. The lambda instructions for
these are \lstinline!Lstaticraise! and \lstinline!Lstaticcatch!.

In this case, we can implement these using the C \lstinline!goto! feature,
which allows control flow to jump non-linearly to another point in the same
function.

The structure gets compiled into the following construct:

\begin{lstlisting}
if (1) {
    /* body */
    // goto label_staticcatch_1;
} else {
label_staticcatch_1:;
    /* handler */
}
\end{lstlisting}

\subsection{Non-static exceptions}

More generally an exception may unwind through several stack frames, and may be
caught in different handlers depending on the dynamic runtime behaviour.

C has a mechanism to perform ``non-local'' jumps using the special library
calls \lstinline!setjmp! and \lstinline!longjmp!. A call to \lstinline!setjmp!
will save the CPU registers at the point the function was called (notably,
including the instruction and stack pointers) into a \lstinline!struct jmpbuf!
that the user provides. Calling \lstinline!longjmp! will then restore the state
of the program, so that execution flow resumes as if the original
\lstinline!setjmp! returned for a second time. \lstinline!setjmp! will return
non-zero iff it is returning for the second time.

I use this mechanism to support exception handling. TODO

\begin{lstlisting}
  if (setjmp(jmp_buf) == 0) {
    // first time
    longjmp(jmp_buf);
  } else {
    // second time
  }
\end{lstlisting}

\lstinline!setjmp! allows control flow to ``unwind'' to the last installed
exception handler. The value of the exception currently being handled is stored
in a global variable known to the compiler, so that exception handlers can
pattern match against it. The exception handler is uninstalled (``popped'' off
the stack of exception handlers) when:
\begin{itemize}
  \item a try block finishes without raising; or
  \item a handler starts executing. This prevents exceptions raised inside the
      handler from incorrectly being recursively handled by itself.
\end{itemize}

The memory representation of an exception contains the stringified name of the
exception, so that if it is propagated all the way up unhandled, the runtime
can print the name of the exception before aborting.

\subsection{Runtime support}

\subsubsection{Toplevel handler}

If an exception propagates all the way to the top without being handled
successfully, the runtime needs to print an error message and abort the
program. This is done by installing a root handler in \lstinline!main! before
calling into OCaml code.

Assuming that the module being compiled is called \lstinline!Test!, the
following is the C \lstinline!main! function:

\begin{lstlisting}
void Test__init();

int main() {
    // set up root exception handler
    OCAML_LIBALLOCS_EXN_PUSH();
    if (0 == OCAML_LIBALLOCS_EXN_SETJMP()) {
        Test__init();
        return 0;
    } else { // catch
        OCAML_LIBALLOCS_EXN_POP();
        fprintf(stderr, "Uncaught OCaml exception: %s\n",
                (const char *)
                  GET_P(GET_P(ocaml_liballocs_get_exn())[0]));
        return 1;
    }
}
\end{lstlisting}

\subsubsection{Builtin exceptions}\label{builtin-exceptions}

A number of important exceptions are embedded fairly deep in the OCaml
language, surprisingly at a lower level than even the \lstinline!Pervasives!
module. OCaml raises some of these exceptions specially for various reasons --
for example the \lstinline!Division_by_zero! OCaml exception is raised when the
corresponding FPU exception occurs.

As these exceptions are not defined in the usual way, in OCaml code, we must
manually declare and instantiate these exceptions in our own C runtime library.
The list of predefined exceptions can be found in \lstinline!typing/predef.ml!.
The C preprocessor quote trick allows us to succinctly define the list of
builtin exceptions statically:

\begin{lstlisting}
#define _QUOTE(x) #x
#define QUOTE(x) _QUOTE(x)
#define DEFINE_BUILTIN_EXCEPTION(name) \
    ocaml_value_t __##name[2] = { \
        NEW_P((generic_datap_t) QUOTE(name)), \
        NEW_I(0) \
    }; \
    ocaml_value_t *name = __##name;

DEFINE_BUILTIN_EXCEPTION(Match_failure)
DEFINE_BUILTIN_EXCEPTION(Assert_failure)
...
\end{lstlisting}

\section{Closures}\label{closures}

C allows function references to be stored in variables, in the form of
``function pointers'' containing the address of the start of the function's
machine code. Calls are performed on function pointers by an indirect jump to
the address (using the \lstinline!call! instruction).

The C language has no notion of closures, which OCaml requires to provide
support for lexical scoping of variables in first-class functions. Closures are
conceptually a tuple of the function pointer and an environment, which stores
the free variables of the closure. Hence, we need to add support for generating
closures manually.

One way to represent closures is as ``fat pointers'', where each closure is
stored as a tuple of two pointers. Note however that in OCaml a closure may be
used whenever a function is expected, so each indirect function call would now
have to check whether the function reference is fat or not -- a significant
runtime cost.

A unified way to call closures and function pointers is desired. In
particular, we'd like the act of closing a C function \lstinline!f_impl! with
an environment \lstinline!env! to produce a fresh function pointer
\lstinline!f_closure!, such that calling
\lstinline!f_closure(1, 2, 3)!
with the standard C calling convention is equivalent to a call to
\lstinline!f_impl(1, 2, 3, env)!.

The technique that we use involves the creation of machine code stubs at
runtime. This was pioneered by Breuel (TODO: cite). However, our implementation
was developed completely from scratch, and many details differ from the
approach outlined in the paper.

\subsection{Runtime support}\label{closures-runtime-support}

An internal C function \lstinline{ocaml_liballocs_close(f_impl, n_args, env)}
was written, which performs the closing operation as previously described.
\lstinline!f_impl! is the C implementation of the closure -- a function which
takes \lstinline!n_args! arguments along with an extra argument containing the environment.

When \lstinline!ocaml_liballocs_close! is
invoked, a small executable stub is written into an executable
buffer, with the two pointers \lstinline!f_impl! and \lstinline!env! baked in.
Then the address of the start of this stub is returned. The stub is described
in detail below.

Conceptually, the operation can be described as JIT-compiling the following
snippet of pseudo-C code at runtime. (NB: things are a bit more complicated
when \lstinline!n_args!${} > 5$ arguments, as described in the sections below.)

\begin{lstlisting}
ocaml_value_t f_closure(ocaml_value_t arg1,
                        ...,
                        ocaml_value_t argn)
{
    return f_impl(arg1, ..., argn, env);
}
\end{lstlisting}

Note that as implemented, this won't work on anything but 64-bit Linux.

Note that this code must be written to a memory page which has both write and
execute permissions. (This has security implications which mean that additional
precautionary steps must be taken if this technique is to be used in production
code.)

This area of memory is bump allocated. Note that a garbage collector would need
to be taught about this.

\subsubsection{Closing functions that take $\le 5$ arguments}

An Application Binary Interface (ABI) specifies a standardised calling
convention which programs and libraries adhere to. This includes the
specification of the location in registers or memory where function arguments
are passed.

On POSIX compliant systems (this does not include Windows!) C uses AMD's x86-64
ABI. The first six arguments are passed in registers in a particular order,
shown in the table below. Hence, when closing a function \lstinline!f! which
originally took five or fewer arguments, there is space to pass \lstinline!env!
in another register.

Conveniently, all six argument registers are also caller-preserved whether they
are used to pass arguments or not. This means that our stub is allowed to
overwrite (or ``clobber'') any of them without having to clean up afterwards.
There are also several other caller-preserved registers, notably
\lstinline!r10!, which we will also make use of.

The following is the table of the register used for each argument in the x86-64
ABI.

\begin{tabular}{ c | c }
  $n$th argument & register \\
  \hline
  1 & \lstinline!rdi! \\
  2 & \lstinline!rsi! \\
  3 & \lstinline!rdx! \\
  4 & \lstinline!rcx! \\
  5 & \lstinline!r8! \\
  6 & \lstinline!r9!
\end{tabular}

We shall place \lstinline!env! into the $(\text{\lstinline|n\_args|}+1)$-th
argument. Call the corresponding register \lstinline!REG_ENV!.

This is the assembly listing for the machine code generated:

\begin{lstlisting}
mov REG_ENV, <env>
mov r10, <f_impl>
jmp r10
\end{lstlisting}

This works by using two 8-byte immediate \lstinline!mov! instructions, which
each load a fixed constant from the program code, which happen to correspond to
our two pointers \lstinline!env! and \lstinline!f_impl!.
Then \lstinline!f_impl! is tail-called into.

\subsubsection{Closing functions that take $> 5$ arguments}

When \lstinline!n_args! $> 5$, \lstinline!env! needs to get passed in on the
stack instead.

\begin{tabular}{c}
  return address to caller
  \\ \hline\hline
  \lstinline!arg_7!
  \\ \hline
  \vdots
  \\ \hline
  \lstinline!arg_n!
\end{tabular}

However, implementing the direct solution is a little awkward because:
\begin{enumerate}
  \item when we modify the stack, our stub can no longer just tail-call
    into \lstinline!f_impl! -- before returning to the caller we need to undo
    our stack transformation;
  \item we need to tuck \lstinline!env! in under the return address
    stack slot, which would require a large number of stack-twiddling
    instructions.
\end{enumerate}

Luckily, I came up with a neat trick which addresses both issues and remains
performant. It turns out that we can get away with just pushing \lstinline!env!
on top of the stack as follows, before \lstinline!call!ing into
\lstinline!f_impl!. The state of the stack on entry to \lstinline!f_impl! is:

\begin{tabular}{c}
  return address to stub
  \\ \hline\hline
  \lstinline!env!
  \\ \hline
  return address to caller
  \\ \hline
  \lstinline!arg_7!
  \\ \hline
  \vdots
  \\ \hline
  \lstinline!arg_n!
\end{tabular}

By treating the return address to caller as an extra (unused) argument to
\lstinline!f_impl!, it can now access \lstinline!env! through its
(\lstinline{n_args}${}+2$)th argument.

This trick works because our compiler has control over the function signature
of \lstinline{f} -- in this case it'll just generate the following signature:

\begin{lstlisting}
ocaml_value_t f(ocaml_value_t arg_1,
                ...
                ocaml_value_t arg_6,
                ocaml_value_t *env,
                void *unused,
                ocaml_value_t arg_7,
                ...
                ocaml_value_t arg_n);
\end{lstlisting}

Now the machine code stub looks like this:

\begin{lstlisting}
mov r11, <env>
push r11
mov r10, <f_impl>
call r10
pop rcx
ret
\end{lstlisting}

Note the deliberate choice of \lstinline{pop rcx} to remove the
\lstinline{env} address from the stack (instead of, for instance
\lstinline{pop r11} or \lstinline{add rsp, 8}). This is a size optimisation --
the chosen instruction encodes in just one byte.

\subsection{Compiler support}\label{closures-compiler-support}

As OCaml immediate value representations are immutable, we can simply copy the
values of each free variable into the environment at the time of closure
creation.

There is no explicit annotation in the OCaml lambda IR to indicate that a
closure must be created -- the compiler has to perform free-variable analysis
to determine this itself. We handle three separate scenarios where closures
must be created:

\begin{itemize}
    \item \textbf{Function let-binding with non-empty free variable set}:
      \begin{lstlisting}
let make_counter_v1 () =
  let ctr = ref 0 in
  let count () = ctr := !ctr + 1; !ctr in
  count
      \end{lstlisting}
    \item \textbf{Anonymous lambda with non-empty free variable set}:
      \begin{lstlisting}
let make_counter_v2 () =
  let ctr = ref 0 in
  fun () -> (ctr := !ctr + 1; !ctr)
      \end{lstlisting}
    \item \textbf{Partial application}:
      \begin{lstlisting}
let make_counter_v3 () =
  let count ctr () = ctr := !ctr + 1; !ctr in
  count (ref 0)
      \end{lstlisting}
\end{itemize}

TODO: explain what we have to do with these

\subsubsection{Recursive and mutually recursive closures}

Mutually recursive closures share an environment. Hence calls between two
mutually recursive closures must pass the same environment pointer to each
invocation.

Self-recursive closures are a special case of this.



\chapter{Evaluation}

\section{Compiling the OCaml \texttt{List} module}\label{module-list}

As a demonstration of the capabilities of my OCaml compiler, I am able to
directly compile the \lstinline!List! module from the upstream standard
library, with no modifications.

Although many of the functions in the module are very basic (e.g.
\lstinline!hd!), others serve as good demonstrations of recursion, first order
functions and closures, such as \lstinline!filter!, \lstinline!fold_left!. By
far the most complicated function in the module is \lstinline!sort!, which
implements an optimised mergesort.

Compiling the \lstinline!List! module is not just for demonstration purposes,
however. It is a required step in building the compiler, as many benchmarks
rely on this module. The way that \lstinline!List! support is provided to user
code is during tthe linking stage: the user's generated C code is linked
against the generated C code of the \lstinline!Pervasives! and \lstinline!List!
modules.

\section{Regression testing}

As new features have been implemented in the compiler, I've been writing small
tests which ensure the correct functionality of that part of the compiler. In
total 22 separate test files were written, most of which test multiple
variations in a single file. Each commit to the master branch strives to pass
all unit tests\footnote{At the end of my project, there were two expected
  failures: overapplication and submodule function arity inference, as
  described in section \ref{conclusion}}
-- works in progress were done on a separate branch before merging into master
when ready.

A selection of illustrative tests, and the C code produced, is given below.

\begin{enumerate}
    \item \textbf{\texttt{arithmetic}}: tests integer arithmetic and shift
        operations, and floating point operations. In particular the sign
        extension behaviour on 50-bit integers is carefully tested for both
        arithmetic shift and logical shift operations -- as described in
        section \ref{50-bit-integer};
    \item \textbf{\texttt{basic\char`_ops}}: tests compilation and behaviour of
        nested if statements -- section \ref{module-fixup};
    \item \textbf{\texttt{basic\char`_types}}: tests basic types: integers,
        booleans, tuples, lists, reference types
    \item \textbf{\texttt{closure}}: tests the first two styles of closures
        (non-empty free variable set) -- as described in section
        \ref{closures};
    \item \textbf{\texttt{partialapp}}: tests the last style of closure
        (partial application);
    \item \textbf{\texttt{closure\char`_partialapp}}: tests that partial
        application of a closure itself creates a working closure (contrast
        with the partialapp test);
    \item \textbf{\texttt{closure\char`_letrec}}: tests mutually recursive
        closures;
    \item \textbf{\texttt{closure\char`_toplevel}}: ensures that functions with
        free variables that are globally scoped do not needlessly get
        transformed into closures;
    \item \textbf{\texttt{exn}}: tests raising and catching exceptions with and
        without parameters -- section \ref{exceptions};
    \item \textbf{\texttt{exn\char`_2}}: ensure that exception handlers are
        popped correctly;
    \item \textbf{\texttt{exn\char`_builtin}}: tests builtin exceptions --
        section \ref{builtin-exceptions};
    \item \textbf{\texttt{exn\char`_fatal}}: tests that uncaught exceptions
        trigger a message and abort the program;
    \item \textbf{\texttt{while}}: tests imperative-style while loops --
        section \ref{while-loops}.
    \item \textbf{\texttt{for}}: tests range-based for loops, including ranges
        that are negative-lengthed -- section \ref{for-loops};
    \item \textbf{\texttt{letrec}}: tests basic mutually recursive functions --
        section \ref{mutually-recursive-functions};
    \item \textbf{\texttt{letrec\char`_value}}: tests the ``mutually recursive
        values'' OCaml feature -- section \ref{mutually-recursive-values};
    \item \textbf{\texttt{match}}: tests that pattern matching compiles
        correctly;
    \item \textbf{\texttt{print}}: tests standard library printing functions --
        section \ref{pervasives-printing};
    \item \textbf{\texttt{stdlib\char`_list}}: tests standard library
        \lstinline!List! functions -- section \ref{module-list};
    \item \textbf{\texttt{stringswitch}}: tests pattern matching on strings --
        section \ref{stringswitch};

--

    \item \textbf{\texttt{1}}: TODO
    \item \textbf{\texttt{overapplication}}: TODO
\end{enumerate}

\section{Benchmarks}

My benchmarking methodology relies on a shell script \lstinline!benchmark.sh!,
which runs twelve versions of each benchmark:

\begin{itemize}
    \item Bytecode generated by the upstream \lstinline!ocamlc! bytecode compiler. These are run with \lstinline!ocamlrun! (the upstream optimised bytecode interpreter), with default GC settings, and disabled GC settings;
    \item Native code generated by the upstream \lstinline!ocamlopt! native compiler, with default GC settings, and disabled GC settings;
    \item Native code generated by \lstinline!gcc! via my OCaml-to-C compiler, with each of four optimisation levels (\lstinline!-O0!, \lstinline!-O1!, \lstinline!-O2!, \lstinline!-O3!);
    \item Native code generated by \lstinline!clang! via my OCaml-to-C compiler, with each of four optimisation levels.
\end{itemize}

My benchmarks broadly fall into one of three categories.

\subsection{Microbenchmarks}
Microbenchmarks written by me. These test particular aspects of the
performance of the generated code.

\begin{enumerate}
  \item
    The \textbf{\texttt{closure\char`_create\char`_<n>}} family of benchmarks test the performance of
    creating closures which take $n$ arguments and capture 1 environment value.

  \item
    The \textbf{\texttt{closure\char`_invoke\char`_<n>}} family of benchmarks the performance of
    invoking closures which take $n$ arguments and capture 1 environment value.

  \item
    The \textbf{\texttt{closure\char`_capture\char`_<n>}} family of benchmarks the performance
    of invoking closures which take 1 argument and capture $n$ variables from their environment.

  \item

  The \textbf{\texttt{list\char`_sort}} benchmark was written from scratch, to test the
  performance of the unmodified pure-OCaml standard library \lstinline!List.sort!
  routine (see section \ref{module-list}).

  The routine only creates a constant number of closures, but makes heavy use of
  closure invocation and allocates heavily due to the list operations it
  performs.
\end{enumerate}

\subsection{\texttt{operf-micro} benchmark suite}
These are various sized benchmarks adapted from the OCamlPro's
      \lstinline!operf-micro! benchmark suite. These needed adapting before
      they could be used, due to being dependent on an OCaml-side framework
      which performs timing and measures GC statistics, features which are not
      supported by my compiler.
\begin{enumerate}

  \item
    The \textbf{\texttt{fibonacci}} benchmark was adapted from the \lstinline!fibonnaci!
[\textit{sic}] benchmark in \lstinline!operf-micro!.

This simple microbenchmark implements the naive (non-memoising) recursive
Fibonacci function, and uses it to evaluate the $40$-th Fibonacci number.

The benchmark primarily stress-tests the speed of function calls, as the cost
of invoking recursive calls dominates the amount of work done inside each call.

\item

  The \textbf{\texttt{lens}} benchmark implements a Haskell-style lens (otherwise known as
functional references), and then uses lenses to operate on a
\lstinline!rectangle! record type.

This makes usage of records and higher order functions, creating a very large
number of closures.

\item

  The \textbf{\texttt{sieve}} benchmark implements a linked-list sieve of Eratosthenes.

This benchmark performs many list operations and is very allocation-heavy.

\item

  The \textbf{\texttt{vector\char`_functor}} benchmark implements 2D and 3D vectors in two
different styles -- directly and by using a generic functor. It then uses these
implementations to create and calculate the dot product of $10^6$ 2D and 3D
vectors.

This benchmark allocates many records, and performs floating point operations.
\end{enumerate}

\subsection{Macrobenchmarks}
These are larger benchmarks written by me, as solutions to Project Euler problems.

\begin{enumerate}
\item

The \textbf{\texttt{p153}} benchmark implements a solution to Project Euler problem 153.

This makes usage of exceptions to break out of a recursive loop early.

\item
  The \textbf{\texttt{p156}} benchmark implements a solution to Project Euler problem 156.

This makes heavy use of recursion and imperative-style \lstinline!for! loops.

\item
  The \textbf{\texttt{p182}} benchmark implements a solution to Project Euler problem 182.

This makes use of a tail-recursive GCD implementation, a counter reference and imperative-style \lstinline!for! loops.

\item
  The \textbf{\texttt{p183}} benchmark implements a solution to Project Euler problem 183.

This makes heavy use of both integer and floating point arithmetic (\lstinline!log!, \lstinline!exp!, floating point division).
\end{enumerate}

\chapter{Conclusion}\label{conclusion}

Limitations:

\begin{itemize}
  \item No tag in blocks. This means no support for variants
  \item ``Overapplication'' uncurrying
  \item Large parts of the standard library are unimplemented. Significantly \lstinline!Array!
  \item Inter-module function arity is unknown -- cannot partially apply or overapply (TODO: is this true?)
\end{itemize}
%TODO


%%%%%%%%%%%%%%%%%%%%%%%%%%%%%%%%%%%%%%%%%%%%%%%%%%%%%%%%%%%%%%%%%%%%%
% the bibliography
\addcontentsline{toc}{chapter}{Bibliography}
\bibliography{refs}

%%%%%%%%%%%%%%%%%%%%%%%%%%%%%%%%%%%%%%%%%%%%%%%%%%%%%%%%%%%%%%%%%%%%%
% the appendices
\appendix

%\chapter{Latex source}
%
%\section{diss.tex}
%{\scriptsize\verbatiminput{diss.tex}}
%
%\section{proposal.tex}
%{\scriptsize\verbatiminput{proposal.tex}}
%
%\chapter{Makefile}
%
%\section{makefile}\label{makefile}
%%{\scriptsize\verbatiminput{makefile.txt}}
%
%%\section{refs.bib}
%%{\scriptsize\verbatiminput{refs.bib}}
%
%
\chapter{Project Proposal}

%\documentclass[12pt,twoside,a4paper]{article}

\usepackage[pdfborder={0 0 0}]{hyperref}
\usepackage[margin=25mm]{geometry}
\usepackage{parskip}

\usepackage{a4}             % Adjust margins for A4 media

\usepackage{lastpage}       % "n of m" page numbering
\usepackage{lscape}         % Makes landscape easier
%\usepackage{portland}       % Switch between portrait and landscape
\usepackage[usenames,dvipsnames,svgnames,table]{xcolor}         % X11 colour names
\usepackage{graphics}       % Graphics commands
\usepackage{wrapfig}        % Wrapping text around figures
\usepackage{epsfig}         % Embed encapsulated postscript
\usepackage{rotating}       % Extra graphics rotation
%\usepackage{tables}         % Tabular environments
\usepackage{longtable}      % Page breaks within tables
\usepackage{supertabular}   % Page breaks within tables
\usepackage{multicol}       % Allows table cells to span cols
\usepackage{multirow}       % Allows table cells to span rows
%\usepackage{texnames}       % Macros for common tex names
%\usepackage{trees}          % Tree-like layout
\usepackage{mdframed}       % frames around paragraphs

\usepackage{listings}       % Source code listings
\usepackage{courier}        % courier font
\lstset{
  basicstyle=\ttfamily
, commentstyle=\color{Green}
, keywordstyle=\bfseries\color{RoyalBlue}
, showspaces=false
, showstringspaces=false
, breaklines=true
, breakatwhitespace=true
, framextopmargin=50pt
, columns=fullflexible,keepspaces
, escapeinside={(*}{*)}
%, frame=bottomline
      }
\usepackage{array}          % Array environment
\usepackage[shortlabels]{enumitem}       % fancy enum settings
\usepackage{url}            % URL formatting
\usepackage{amsmath}        % American Mathematical Society
\usepackage{amssymb}        % Maths symbols
\usepackage{amsthm}         % Theorems
%\usepackage{mathpartir}     % Proofs and inference rules
\usepackage{verbatim}       % Verbatim blocks
\usepackage{fancyvrb}       % Verbatim blocks
\usepackage{ifthen}         % Conditional processing in tex
\usepackage{caption}         % \caption*, no colon

\usepackage{graphicx}
\usepackage{tikz}
\usepackage[export]{adjustbox} %valign=t
\usepackage{siunitx}        % non-slanted units in math mode with \SI{1}{\volt}
\usepackage{nicefrac}          % slanted fractions with \nicefrac{1}{2}

\newcommand*\BitAnd{\mathrel{\&}}
\newcommand*\BitOr{\mathrel{|}}
\newcommand*\ShiftLeft{\ll}
\newcommand*\ShiftRight{\gg}
\newcommand*\BitNeg{\ensuremath{\mathord{\sim}}}
\usepackage{mathtools}
\DeclarePairedDelimiter{\ceil}{\lceil}{\rceil}
\mathtoolsset{showonlyrefs=true} % show only eq refs that are used

\usepackage{float}

\renewcommand{\thispagestyle}[1]{}

\setlength{\parindent}{0em}

\begin{document}

\begin{center}
\Large Computer Science Tripos -- Part II -- Project Proposal
\\[4mm]
\LARGE \bfseries An observable OCaml, via C and \lstinline{liballocs} \mdseries
\\[4mm]

\large
Cheng Sun, Churchill College

Originator: Stephen Kell

\makeatletter
\@date
\makeatother

\end{center}

\vspace{5mm}

\textbf{Project Supervisor:} Stephen Kell

\textbf{Director of Studies:} John Fawcett

\textbf{Project Overseers:} Timothy Griffin \& Pietro Lio


\section*{Introduction}

OCaml is one of the most commonly used members of the ML family of functional languages. It is popular for its expressivity, type system and performance. However, there is not as of yet a good story for debugging OCaml programs, and observing their behaviour at runtime.

The OCaml bytecode debugger, \lstinline{ocamldebug}, forms part of the core OCaml toolchain. One major problem is that \lstinline{ocamldebug} is unable to ``see through'' polymorphism. For instance, suppose we would like to debug the following polymorphic list-reverse function, which has type \lstinline{'a list -> 'a list}:

\begin{lstlisting}
let my_rev lst =
  match lst with
  | [] -> []
  | x::xs -> List.append xs [x]

let result = my_rev [1; 2; 3]
\end{lstlisting}

Now let's try to debug the function with \lstinline{ocamldebug}.

\begin{lstlisting}
$ (!\aftergroup\bfseries!)ocamlc -g -o my_rev my_rev.ml(!\aftergroup\mdseries*)
$ (!\aftergroup\bfseries!)ocamldebug my_rev(!\aftergroup\mdseries!)
        OCaml Debugger version 4.02.3

(ocd) (!\aftergroup\bfseries!)break @ My_rev 1(!\aftergroup\mdseries!)
Loading program... done.
Breakpoint 1 at 21600: file my_rev.ml, line 2, characters 3-62
(ocd) (!\aftergroup\bfseries!)run(!\aftergroup\mdseries!)
Time: 12 - pc: 21600 - module My_rev
Breakpoint: 1
2   <|b|>match lst with
(ocd) (!\aftergroup\bfseries!)print lst(!\aftergroup\mdseries!)
lst: 'a list = [<poly>; <poly>; <poly>]
\end{lstlisting}

Note that \lstinline{ocamldebug} is unable to display the contents of the input list, as it does not know the concrete type that the type variable \lstinline{'a} is instantiated with for this invocation.

There are many other deficiencies with the OCaml debugger, mostly stemming from its immaturity and lack of features. These issues mean that when debugging OCaml code, one often has to resort to ``printf debugging'' instead.

The aim of this project is to investigate whether the experience of debugging and observing the runtime behaviour of OCaml programs could be improved by utilising the mature C toolchain. My goal is to write a translator that compiles OCaml code into equivalent C code, whilst maintaining a well-defined mapping between the two (in terms of variable names, types, and so on). A user will then be able to debug their OCaml program by using (perhaps an augmented) \lstinline{gdb} on the generated C code.

In order to solve the problem of ``seeing through'' polymorphism, I will use a library written by my supervisor, \lstinline{liballocs}, which keeps track of runtime allocation metadata (including their types) with low overhead. This will allow the debugger to inspect the allocation corresponding to the input to \lstinline{my_rev}, for instance, and conclude that it is operating on lists with elements of type \lstinline{int}.


\section*{Starting point}

The project will be focused on creating a new ``backend'' for the OCaml
compiler, so I will build on the existing code of the frontend.
This includes reusing code for lexing, parsing and typing of OCaml programs, as
well as the transformation from the typed AST to the Intermediate
Representation that we will be using as input.

In order to provide a standard library to compile programs against, I will try
to use as much of the existing OCaml \lstinline{stdlib} as possible. If this
turns out not to be feasible then I will resort to writing a subset of the
standard library by hand.

I will use the open-source library \lstinline{liballocs}, written by Stephen Kell.
This library provides routines to tag regions in virtual memory
(``allocations'') with run-time type information.

I may also make use of some further open-source libraries such as:
\begin{itemize}
  \item \lstinline{libffi} -- a library providing a portable way to perform a call to a function
with a foreign function interface (such as the closures that I will dynamically
create at runtime);
  \item \lstinline{cil} -- a library written in OCaml providing a framework for the manipulation of C programs.
\end{itemize}

There exists similar prior work \cite{tarditi90} that shows that the overall project concept is feasible. However, I will be making different design choices to the compiler presented in the paper, due to
differing requirements. For instance, the presented compiler
generates code that uses continuation-passing style, which I would like to
avoid doing, as it harms observability -- backtraces would no longer be directly
meaningful.


\section*{Resources required}

The project will primarily be developed on my personal computer
for convenience. The computer has a 4-core 3.30GHz Intel Xeon
CPU, a 250GB SSD and 8GB of RAM. I am using Arch Linux. No other non-standard equipment
is anticipated to be required, and I can easily continue development on
MCS machines if required.

The project will be synced to a git server (likely GitHub), and backed up
regularly to my MCS network drive.

The project will utilise various standard open-source software packages,
including the usual C toolchain (e.g. \lstinline{gcc}, \lstinline{gdb}, \lstinline{make}) along with
the OCaml compiler. (As mentioned previously, \lstinline{liballocs} requires
some patches to \lstinline{binutils}, but these can easily be built.)

\section*{Work to be done}

The project can be split up into the following tasks:

\begin{enumerate}
  \item
    Performing a study of the starting point (OCaml compiler frontend and \lstinline{liballocs}).
    This includes an investigation into the forms of the various Intermediate Representations of the OCaml compiler, and selection of the best IR to use as input for our project.
  \item
    Choosing a suitable object representation for boxed objects in memory. There are a variety of alternatives to OCaml's own tagged-pointer representation (which leads to awkward 63-bit integers).
    This choice requires consideration of the garbage collection strategy, even if a GC is not within the core scope of this project.
  \item
    Implementing the translation from OCaml IR to C (with \lstinline{liballocs}), for a variety of language features:
    \begin{enumerate}
      \item Fundamental types: \lstinline{bool}, \lstinline{int}, \lstinline{float}, \lstinline{list}, \lstinline{ref}, \ldots;
      \item Tuples and records, which can be implemented using C \lstinline{struct}s;
      \item (Non-polymorphic) variants, which can be implemented using tagged C \lstinline{union}s;
      \item Polymorphic types, which can be represented safely using \lstinline{void *}
        opaque pointers, as downcasts are never required
        in ML (although \lstinline{liballocs} could provide the necessary information);
      \item Parametrically polymorphic functions;
      \item First class functions and closures, which require lambda lifting,
        or more generally something like the technique of Breuel
        \cite{breuel88} (dynamically creating an executable stub associated
        with each closure at runtime).
    \end{enumerate}
  \item
    Integrating a subset of the OCaml \lstinline{stdlib}, or writing my own as necessary.
  \item
    Writing a runtime library to support dynamic closure creation.
  \item
    Creating a corpus of test and benchmark programs.
  \item
    Evaluation, as described in the \textit{Success criteria} section below.
  \item
    Extensions as time permits, as described in the \textit{Possible extensions} section below.
\end{enumerate}

\section*{Success criteria}

My project will have been successful if I have managed to create a OCaml-to-C translator that works on a commonly used subset of the OCaml language (including closures, parametrically polymorphic functions). Furthermore the polymorphism must be able to be seen through (solving the problem highlighted in the \textit{Introduction} section, for instance).

The project will be evaluated in two ways. Both will require a corpus of OCaml tests and micro-benchmarks.

Firstly, I will measure the ``observability'' of the compiled C programs compared with their native and bytecode OCaml counterparts. For instance, one metric would be obtained by interrupting these programs at pre-determined points in the code, and counting the number of local variables that could be recovered from the stack frames. This is an accurate metric for observability, as the utility of debugging is determined almost completely by the amount of state that is visible at a breakpoint.

Secondly, I will measure the performance of the resultant C executables, by timing benchmark programs as compiled by both compilers. As the primary objective of the project is debuggability rather than performance, I do not expect the performance to be anywhere near as good as code generated by the OCaml native-code compiler: I will be satisfied as long as the performance is within a reasonable factor for the majority of use cases. It is still necessary to evaluate performance, because it affects the debuggability of CPU-intensive programs.

There are many other qualitative evaluations that I will also consider, such as the ease of interoperability with C libraries using this approach, the ability to insert instrumentation into the code, and the overall debugging experience when using \lstinline{gdb} on the resultant executables.

\section*{Possible extensions}

\begin{itemize}
  \item Improving the runtime, such as adding a garbage collector;
  \item Adding further language and standard library support, to increase the range of programs supported;
  \item Optimising the performance of the generated program;
  \item Improving the debugging experience (such as augmenting \lstinline{gdb}).
\end{itemize}

\section*{Timetable}

\begin{enumerate}
\item \textbf{Michaelmas weeks 2--3:}
  Dig into the OCaml compiler frontend; study its Intermediate Representations. Learn to use the \lstinline{liballocs} library.

\item \textbf{Michaelmas weeks 4--6:}
  Begin work on translator, initially targeting basic features (fundamental types, monomorphic functions).
  Design an initial representation for boxed objects.

\item \textbf{Michaelmas weeks 7--8:}
  Add support for tuples, variants, records. Test against very simple programs.

\item \textbf{Michaelmas vacation weeks 1--2:}
  Add support for polymorphic functions and types.

\item \textbf{Michaelmas vacation weeks 3--4:}
  Get a small, commonly-used subset of the standard library working.
  Start creating corpus of test programs, and begin testing.

\item \textbf{Michaelmas vacation weeks 5--6:}
  Reserved for holidays/revision.

\item \textbf{Lent term weeks 0--1:}
  Add initial support for dynamic closure creation.
  Complete corpus of test OCaml programs; begin evaluation.
  Start progress report.

\item \textbf{Lent term week 2--3:}
  Milestone: compiler has working support for most test programs (demoable).
  Submit progress report.
  Prepare and give presentation.

\item \textbf{Lent term weeks 4--6:}
  Testing; fix bugs. Improve closure creation, and evaluate.
  Investigate ``source maps'' to map C code to corresponding lines in the OCaml source.

\item \textbf{Lent term weeks 7--8:}
  Further evaluation, and work on improving evaluated metrics.
  Start on main chapters of dissertation.

\item \textbf{Easter vacation weeks 1--3:}
  Milestone: compiler works for all intended features.
  Wrap up evaluation. Continued work on dissertation.

\item \textbf{Easter vacation weeks 4--5:}
  Reserved for holidays/revision. Extension work if time permits.

\item \textbf{Easter term weeks 0--2:}
  Form conclusion and complete dissertation. Time may be limited due to revision.

\item \textbf{Easter term week 3:}
  Proof reading and submission of final dissertation.



\end{enumerate}

\begin{thebibliography}{9}

\bibitem{tarditi90}
  D. Tarditi, A. Acharya, P. Lee,
  \emph{No Assembly Required: Compiling Standard ML to C},
  November 1990. \\ \url{http://repository.cmu.edu/cgi/viewcontent.cgi?article=3011&context=compsci}

\bibitem{breuel88}
  Thomas Breuel,
  \emph{Lexical Closures for C++},
  In Proc. USENIX C++ Conf., pages 293-304,
  Denver, CO, October 1988.  \\ \url{http://www.cl.cam.ac.uk/~srk31/teaching/redist/breuel88lexical.pdf}

\end{thebibliography}

\end{document}


\end{document}
