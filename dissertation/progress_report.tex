\documentclass[12pt,twoside,a4paper]{article}

\usepackage[pdfborder={0 0 0}]{hyperref}
\usepackage[margin=25mm]{geometry}
\usepackage{parskip}

\usepackage{a4}             % Adjust margins for A4 media

\usepackage{lastpage}       % "n of m" page numbering
\usepackage{lscape}         % Makes landscape easier
%\usepackage{portland}       % Switch between portrait and landscape
\usepackage[usenames,dvipsnames,svgnames,table]{xcolor}         % X11 colour names
\usepackage{graphics}       % Graphics commands
\usepackage{wrapfig}        % Wrapping text around figures
\usepackage{epsfig}         % Embed encapsulated postscript
\usepackage{rotating}       % Extra graphics rotation
%\usepackage{tables}         % Tabular environments
\usepackage{longtable}      % Page breaks within tables
\usepackage{supertabular}   % Page breaks within tables
\usepackage{multicol}       % Allows table cells to span cols
\usepackage{multirow}       % Allows table cells to span rows
%\usepackage{texnames}       % Macros for common tex names
%\usepackage{trees}          % Tree-like layout
\usepackage{mdframed}       % frames around paragraphs

\usepackage{listings}       % Source code listings
\usepackage{courier}        % courier font
\lstset{
  basicstyle=\ttfamily
, commentstyle=\color{Green}
, keywordstyle=\bfseries\color{RoyalBlue}
, showspaces=false
, showstringspaces=false
, breaklines=true
, breakatwhitespace=true
, framextopmargin=50pt
, columns=fullflexible,keepspaces
, escapeinside={(*}{*)}
%, frame=bottomline
      }
\usepackage{array}          % Array environment
\usepackage[shortlabels]{enumitem}       % fancy enum settings
\usepackage{url}            % URL formatting
\usepackage{amsmath}        % American Mathematical Society
\usepackage{amssymb}        % Maths symbols
\usepackage{amsthm}         % Theorems
%\usepackage{mathpartir}     % Proofs and inference rules
\usepackage{verbatim}       % Verbatim blocks
\usepackage{fancyvrb}       % Verbatim blocks
\usepackage{ifthen}         % Conditional processing in tex
\usepackage{caption}         % \caption*, no colon

\usepackage{graphicx}
\usepackage{tikz}
\usepackage[export]{adjustbox} %valign=t
\usepackage{siunitx}        % non-slanted units in math mode with \SI{1}{\volt}
\usepackage{nicefrac}          % slanted fractions with \nicefrac{1}{2}

\newcommand*\BitAnd{\mathrel{\&}}
\newcommand*\BitOr{\mathrel{|}}
\newcommand*\ShiftLeft{\ll}
\newcommand*\ShiftRight{\gg}
\newcommand*\BitNeg{\ensuremath{\mathord{\sim}}}
\usepackage{mathtools}
\DeclarePairedDelimiter{\ceil}{\lceil}{\rceil}
\mathtoolsset{showonlyrefs=true} % show only eq refs that are used

\usepackage{float}

\renewcommand{\thispagestyle}[1]{}

\setlength{\parindent}{0em}

\begin{document}

\begin{center}
\Large CST Part II Project -- Progress Report
\\[4mm]
\LARGE \bfseries An observable OCaml, via C and \lstinline{liballocs} \mdseries
\\[4mm]

\large
Cheng Sun \lstinline{<cs799@cam.ac.uk>}

\makeatletter
\@date
\makeatother

\end{center}

\vspace{5mm}

\textbf{Project Supervisor:} Stephen Kell

\textbf{Director of Studies:} John Fawcett

\textbf{Project Overseers:} Timothy Griffin \& Pietro Lio


\section*{Progress}

My project has been progressing well. All of the code I have written so far has been in OCaml, with the exception of portions of the stdlib that the generated C code has to link with. I have completed/made significant progress in the following:

\begin{itemize}
  \item AST datatypes for the C language, representing statements and toplevel definitions.

  \item
Translation module, translating OCaml lambda IR into ``extended'' C AST.

This is now capable of translating basic types like ints, booleans, tuples; references; lists; functions and complete function application; polymorphic functions; linking against and accessing values in other modules.

  \item
Fixup module, taking an ``extended'' C AST to a standard C AST, normalising invalid expressions that are created during translation.

%Every expression in OCaml can be used as a value, for example in constructs such as \lstinline[language=ML]{let x = (if a then b else c)}. However, the same is not valid in C, so this cannot be directly translated to \lstinline[language=C]|int x = if (a) {b} else {c}|. My ``extended'' C AST allows the translation module to generate such constructs (amongst others); it is the role of the recursive fixup pass to eliminate them before code emission.

  \item 
Emitcode module, transforming the C AST into a string representing the C code, and subsequently outputting this as a file.

  \item 
    OCaml compiler driver changes: added a new flag \lstinline{-target-liballocs} which enables my backend.

\item
  Started a prototype standard library, initially supporting \lstinline[language=ML]{Printf.printf}.

\item
Started writing my own corpus of self-contained OCaml files for testing purposes. These generally perform a mathematical computation of some sort, which can be verified/benchmarked.

\end{itemize}

I estimate that I am currently 1--2 weeks behind in my project, due to some personal time management issues last term (outlined below). However, whilst I have been working on it, my project has been progressing very quickly, and I am confident about being able to catch up with where I want to be by the end of this month. Hence, I do not need to make any changes to my project plan.


\section*{Difficulties}

Time allocation during Michaelmas was more difficult than anticipated, and I ended up not starting my project in earnest until the term had ended. However, I managed to catch up rapidly, and the project is in far better shape now. As I am attending fewer courses in Lent, this term will allow more time for me to focus on project work.

It turned out that there was no perfect choice of Intermediate Representation in the compiler which immediately satisfied all of my needs. I ended up choosing to start with the so-called ``lambda'' IR, which is based on a (heavily augmented) untyped lambda calculus. This has been suitable for C translation purposes, as most constructs have a fairly direct mapping. However, I will need type information later in the project, which this IR does not provide. I'm working on extracting this information from typing environments instead.

\end{document}
